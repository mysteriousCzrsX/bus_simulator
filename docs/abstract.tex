\documentclass[../main.tex]{subfiles}
\usepackage[utf8]{inputenc}
\usepackage[T1]{fontenc}
\usepackage{graphicx}
\usepackage{longtable}
\usepackage{wrapfig}
\usepackage{rotating}
\usepackage[normalem]{ulem}
\usepackage{amsmath}
\usepackage{amssymb}
\usepackage{capt-of}
\usepackage{hyperref}
\usepackage{float}
\graphicspath{{../}}
\author{Cezary Wieczorkowski}
\date{\today}
\title{Abstract}
\hypersetup{
 pdfauthor={Cezary Wieczorkowski},
 pdftitle={Abstract},
 pdfkeywords={},
 pdfsubject={},
 pdflang={Polish}}
\begin{document}

\selectlanguage{polish}
\begin{abstract}
Celem projektu było zaprojektowanie oraz uruchomienie zestawu laborarotyjnego umożliwiającego
realizację ćwiczeń z sterowania szyną danych w laboratorium techniki cyfrowej. W ramach projektu dokonano analizy
istniejącego stanowiska pod kątem możliwości rozbudowy oraz poprawy funkcjonalności. Następnie zaprojektowano oraz 
zbudowano prototyp nowego stanowiska uwzględniając wnioski z analizy poprzedniego.
\end{abstract}
\textbf{Słowa kluczowe:} Technika cyfrowa, stanowisko laboratoryjne, emulacja, szyna danych, ALU \\
\textbf{Dziedzina nauki i techniki, zgodne z wymogami OECD:} nauki inżynieryjno--techniczne: automatyka, elektronika, 
elektrotechnika i technologie kosmiczne
\newpage
\selectlanguage{english}
\begin{abstract}
The goal of this project was to design and construct a laboratory set that enables the performance of 
exercises on data bus control in a digital electronics laboratory. As part of the project, an analysis of the existing 
station was carried out in terms of the possibility of expansion and improvement of functionality. Then, a prototype of 
the new station was designed and built, taking into account the conclusions from the analysis of the previous one.
\end{abstract}
\textbf{Keywords:} Digital electronics, laboratory set, emulation, data bus, ALU \\
\selectlanguage{polish}
\end{document}
