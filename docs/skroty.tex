
\documentclass[./main.tex]{subfiles}
\usepackage[utf8]{inputenc}
\usepackage[T1]{fontenc}
\usepackage{graphicx}
\usepackage{longtable}
\usepackage{wrapfig}
\usepackage{rotating}
\usepackage[normalem]{ulem}
\usepackage{amsmath}
\usepackage{amssymb}
\usepackage{capt-of}
\usepackage{hyperref}
\usepackage{float}
\graphicspath{{../}}
\author{Cezary Wieczorkowski}
\date{\today}
\title{Skróty}
\hypersetup{
 pdfauthor={Cezary Wieczorkowski},
 pdftitle={Skróty},
 pdfkeywords={},
 pdfsubject={},
 pdflang={Polish}}
\begin{document}

\section*{Wykaz ważniejszych oznaczeń i skrótów}
\addcontentsline{toc}{section}{Wykaz ważniejszych oznaczeń i skrótów}

\begin{tabular}{l p{.8\textwidth}}
ALU & (ang. arithmetic logic unit) jednostka arytmetyczno-logiczna\\[0pt]
% todo add some more needed acronyms
\end{tabular}
\end{document}
