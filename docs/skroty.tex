
\documentclass[./main.tex]{subfiles}
\usepackage[utf8]{inputenc}
\usepackage[T1]{fontenc}
\usepackage{graphicx}
\usepackage{longtable}
\usepackage{wrapfig}
\usepackage{rotating}
\usepackage[normalem]{ulem}
\usepackage{amsmath}
\usepackage{amssymb}
\usepackage{capt-of}
\usepackage{hyperref}
\usepackage{float}
\graphicspath{{../}}
\author{Cezary Wieczorkowski}
\date{\today}
\title{Skróty}
\hypersetup{
 pdfauthor={Cezary Wieczorkowski},
 pdftitle={Skróty},
 pdfkeywords={},
 pdfsubject={},
 pdflang={Polish}}
\begin{document}

\section*{Wykaz ważniejszych oznaczeń i skrótów}
\addcontentsline{toc}{section}{Wykaz ważniejszych oznaczeń i skrótów}

\begin{tabular}{l p{.8\textwidth}}
ALU & (ang. arithmetic logic unit) jednostka arytmetyczno-logiczna\\[0pt]
RAM & (ang. random access memory) pamięć o dostępie swobodnym\\[0pt]
LED & (ang. light emitting diode) dioda elektroluminescencyjna\\[0pt]
LCD & (ang. liquid crystal display) wyświetlacz ciekłokrystaliczny\\[0pt]
GPIO & (ang. general-purpose input/output) uniwersalne wejście/wyjście\\[0pt]
I/O & (ang. input/output) wejście/wyjście\\[0pt]
I2C & (ang. inter integrated circuit) szeregowy synchroniczny interfejs komunikacyjny\\[0pt]
SDK & (ang. software development kit) zestaw narzędzi do rozwoju oprogramowania\\[0pt]
\end{tabular}
\end{document}
