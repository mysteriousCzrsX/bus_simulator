\documentclass[../main.tex]{subfiles}
\usepackage[utf8]{inputenc}
\usepackage[T1]{fontenc}
\usepackage{graphicx}
\usepackage{longtable}
\usepackage{wrapfig}
\usepackage{rotating}
\usepackage[normalem]{ulem}
\usepackage{amsmath}
\usepackage{amssymb}
\usepackage{capt-of}
\usepackage{hyperref}
\usepackage{float}
\graphicspath{{../}}
\author{Cezary Wieczorkowski}
\date{\today}
\title{Testy}
\hypersetup{
 pdfauthor={Cezary Wieczorkowski},
 pdftitle={Testy},
 pdfkeywords={},
 pdfsubject={},
 pdflang={Polish}}
\begin{document}


\section{Testy stanowiska}

    \begin{table}[ht]
        \[
        \begin{array}{|c|c|c|c|c|c|}
        \hline
        \textbf{Nadajnik} & \textbf{Odbiornik} & \textbf{ALU} & \textbf{Adres} & \textbf{Rp} \\ \hline
        1 & 0 & 1 & 0 & \textcolor{red}{0} & 0 \\ \hline
        1 & 0 & 0 & 1 & \textcolor{red}{1} & 0 \\ \hline
        0 & 1 & 0 & 0 & \textcolor{red}{2} & 0 \\ \hline
        0 & 1 & 1 & 0 & \textcolor{red}{3} & 0 \\ \hline
        \multicolumn{5}{|c|}{} \\ \hline
        0 & 0 & 0 & 1 & \textcolor{red}{4} & 3 \\ \hline
        0 & 0 & 0 & 1 & \textcolor{red}{5} & 3 \\ \hline
        0 & 1 & 0 & 0 & \textcolor{red}{6} & 3 \\ \hline
        1 & 0 & 1 & 1 & \textcolor{red}{7} & 3 \\ \hline
        \multicolumn{5}{|c|}{} \\ \hline
        0 & 0 & 0 & 1 & \textcolor{red}{8} & 3 \\ \hline
        0 & 0 & 0 & 1 & \textcolor{red}{9} & 3 \\ \hline
        0 & 1 & 1 & 0 & \textcolor{red}{10} & 3 \\ \hline
        1 & 0 & 1 & 1 & \textcolor{red}{11} & 3 \\ \hline
        \end{array}
        \]
        \caption{Zawartość pamięci RAM dla pierwszego programu testowego}
    \end{table}

\end{document}