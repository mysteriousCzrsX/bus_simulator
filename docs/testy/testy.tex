\documentclass[../main.tex]{subfiles}
\usepackage[utf8]{inputenc}
\usepackage[T1]{fontenc}
\usepackage{graphicx}
\usepackage{longtable}
\usepackage{wrapfig}
\usepackage{rotating}
\usepackage[normalem]{ulem}
\usepackage{amsmath}
\usepackage{amssymb}
\usepackage{capt-of}
\usepackage{hyperref}
\usepackage{float}
\usepackage{multirow}
\graphicspath{{../}}
\author{Cezary Wieczorkowski}
\date{\today}
\title{Testy}
\hypersetup{
 pdfauthor={Cezary Wieczorkowski},
 pdftitle={Testy},
 pdfkeywords={},
 pdfsubject={},
 pdflang={Polish}}
\begin{document}


\section{Testy stanowiska}

Testy stanowiska przeprowadzono z wykorzystaniem programów testowych podobnych do programów uruchamianych na zestawie w ramach ćwiczeń laboratoryjnych.
Programy testowe zostały napisane w taki sposób, aby sprawdzić poprawność jak największej ilości elementów zestawu.

\subsection{Program testujący operacje arytmetyczne}

    Program testuje poprawność wykonywania operacji na jednostce arytmetyczno logicznej. Program został celowo
    napisany tak aby zapisywał wartości do wszystkich dostępnych odbiorników w celu przetestowania ich poprawnej implementacji. 
    Program wykonuje następujące operacje:

    \begin{itemize}
        \item Pobranie danych z klawiatury wejściowej do $R_A$.
        \item Skopiowanie do $R_A$ do do $R_B$.
        \item Skopiowanie do $R_A$ do do $R_B$.
        \item Pomnożenie $R_A$ przez 2 i zapisanie wyniku w $R_A$.
        \item Dodanie $R_A$ do $R_B$ i wystawienie wyniku do $R_{wy}$.
    \end{itemize}

    W pierwszym cyklu program pobiera dane z klawiatury wejściowej i zapisuje je do rejestru $R_A$. Następnie kopiuje je do 
    rejestru $R_B$ oraz $R_C$. W kolejnym cyklu zapisuje zawartość $R_A$ do $R_1$ oraz $R_2$, następnie zapisuje wartość na wyjściu
    jednostki ALU do rejestru $R_{wy}$. Zapis do rejestru $R_2$ jest w tym wypadku wykonywany tylko w celu poprawnego ustawienia kodu
    operacji w jednostce ALU, gdyż wykonywana operacja mnoży zawartość rejestru $R_1$ przez 2.
    W ostatnim cyklu program dodaje do siebie zawartości rejestrów $R_A$ i $R_B$ i zapisuje wynik w $R_{wy}$.

    \begin{table}[ht]
        \[
        \begin{array}{|ccc|ccc|cc|c|c|}
        \hline
        \multicolumn{3}{|c|}{\textbf{Nadajnik}} & \multicolumn{3}{|c|}{\textbf{Odbiornik}}
        & \multicolumn{2}{|c|}{\textbf{ALU}} & \textbf{Adres} & \textbf{$R_P$} \\ \hline
        1 & 0 & 1 & 0 & 0 & 1 & 0 & 0 & \textcolor{red}{0000 00} & \multirow{4}{*}{0001} \\ \cline{1-9}
        0 & 0 & 1 & 0 & 1 & 0 & 0 & 0 & \textcolor{red}{0000 01} & \\ \cline{1-9}
        0 & 0 & 1 & 0 & 1 & 1 & 0 & 0 & \textcolor{red}{0000 10} & \\ \cline{1-9}
        0 & 0 & 0 & 0 & 0 & 0 & 0 & 0 & \textcolor{red}{0000 11} & \\ \hline
        \multicolumn{5}{c}{} \\ \hline
        0 & 0 & 0 & 0 & 0 & 0 & 0 & 0 & \textcolor{red}{0001 00} & \multirow{4}{*}{0010} \\ \cline{1-9}
        0 & 0 & 1 & 1 & 0 & 0 & 1 & 0 & \textcolor{red}{0001 01} & \\ \cline{1-9}
        0 & 0 & 1 & 1 & 0 & 1 & 0 & 1 & \textcolor{red}{0001 10} & \\ \cline{1-9}
        1 & 0 & 0 & 0 & 0 & 1 & 0 & 0 & \textcolor{red}{0001 11} & \\ \hline
        \multicolumn{5}{c}{} \\ \hline
        0 & 0 & 0 & 0 & 0 & 0 & 0 & 0 & \textcolor{red}{0010 00} & \multirow{4}{*}{0011} \\ \cline{1-9}
        0 & 0 & 1 & 1 & 0 & 0 & 0 & 1 & \textcolor{red}{0010 01} &  \\ \cline{1-9}
        0 & 1 & 0 & 1 & 0 & 1 & 0 & 1 & \textcolor{red}{0010 10} & \\ \cline{1-9}
        1 & 0 & 0 & 1 & 1 & 0 & 0 & 0 & \textcolor{red}{0010 11} & \\ \hline
        \end{array}
        \]
        \caption{Program testujący operacje arytmetyczne}
    \end{table}

\subsection{Program testujący instrukcje skoku}

    Program testuje poprawność wykonywania skoku pomiędzy instrukcjami. Program wykonuje następujące operacje:

    \begin{itemize}
        \item Pobranie danych z klawiatury wejściowej do $R_A$.
        \item Wykonanie skoku w zależności od wartości $R_A$.
        \item Jeśli $R_A$ = 1 zanegowanie $R_A$ i zapisanie wyniku w $R_{wy}$.
        \item Jeśli $R_A$ = 2 podzielenie $R_A$ przez 2 i zapisanie wyniku w $R_{wy}$.
    \end{itemize}

    W celu wykonania skoku program w ostatniej mikroinstrukcji pierwszego cyklu wykonuje operacje zapisu zawartości rejestru $R_A$ do rejestru $R_I$. 
    Spowoduje to zinterpretowanie wartości w $R_A$ jako czterech najstarszych bitów adresu, czyli adresu kolejnych czterech instrukcji.
    W tym wypadku zakładamy jeśli wartość w rejestrze $R_A$ jest równa 1 program w kolejnym cyklu skoczy do instrukcji o adresie 000100.
    Jeśli wartość w rejestrze $R_A$ jest równa 2 program w kolejnym cyklu skoczy do instrukcji o adresie 001000. Pod tymi rozpoczynają się
    cykle mikroinstrukcji odpowiednio dla zanegowania wartości w $R_A$ i podzielenia wartości w $R_A$ przez 2.

    \begin{table}[ht]
        \[
        \begin{array}{|ccc|ccc|cc|c|c|}
        \hline
        \multicolumn{3}{|c|}{\textbf{Nadajnik}} & \multicolumn{3}{|c|}{\textbf{Odbiornik}}
        & \multicolumn{2}{|c|}{\textbf{ALU}} & \textbf{Adres} & \textbf{$R_P$} \\ \hline
        1 & 0 & 1 & 0 & 0 & 1 & 0 & 0 & \textcolor{red}{0000 00} & \multirow{4}{*}{0011} \\ \cline{1-9}
        0 & 0 & 1 & 0 & 0 & 1 & 0 & 0 & \textcolor{red}{0000 01} & \\ \cline{1-9}
        0 & 0 & 1 & 0 & 0 & 1 & 0 & 0 & \textcolor{red}{0000 10} & \\ \cline{1-9}
        0 & 0 & 1 & 0 & 0 & 0 & 0 & 0 & \textcolor{red}{0000 11} & \\ \hline
        \multicolumn{5}{c}{} \\ \hline
        0 & 0 & 0 & 0 & 0 & 0 & 1 & 0 & \textcolor{red}{0001 00} & \multirow{4}{*}{0011} \\ \cline{1-9}
        0 & 0 & 1 & 1 & 0 & 0 & 0 & 0 & \textcolor{red}{0001 01} & \\ \cline{1-9}
        0 & 0 & 1 & 1 & 0 & 1 & 0 & 0 & \textcolor{red}{0001 10} & \\ \cline{1-9}
        1 & 0 & 0 & 1 & 1 & 0 & 0 & 0 & \textcolor{red}{0001 11} & \\ \hline
        \multicolumn{5}{c}{} \\ \hline
        0 & 0 & 0 & 0 & 0 & 0 & 0 & 1 & \textcolor{red}{0010 0} & \multirow{4}{*}{0011} \\ \cline{1-9}
        0 & 0 & 1 & 1 & 0 & 0 & 0 & 0 & \textcolor{red}{0010 01} &  \\ \cline{1-9}
        0 & 0 & 1 & 1 & 0 & 1 & 0 & 1 & \textcolor{red}{0010 10} & \\ \cline{1-9}
        1 & 0 & 0 & 1 & 1 & 0 & 0 & 0 & \textcolor{red}{0010 11} & \\ \hline
        \end{array}
        \]
        \caption{Program testujący instrukcje skoku}
    \end{table}

    W rozpatrywanym programie wartość w rejestrze $R_P$ jest równa 0011. Dzięki temu po wykonaniu instrukcji warunkowej program 
    powróci do wykonywania dalszych instrukcji. Gdyby wartości w rejestrze $R_P$ rozpoczynały się od 1 i zwiększały o 1  jak w większości 
    przypadków program przy wprowadzeniu wartości 1 wykonał by obie instrukcje warunkowe gdyż druga instrukcja nie zostałaby pominięta.

\end{document}