\documentclass[../main.tex]{subfiles}
\usepackage[utf8]{inputenc}
\usepackage[T1]{fontenc}
\usepackage{graphicx}
\usepackage{longtable}
\usepackage{wrapfig}
\usepackage{rotating}
\usepackage[normalem]{ulem}
\usepackage{amsmath}
\usepackage{amssymb}
\usepackage{capt-of}
\usepackage{hyperref}
\usepackage{float}
\graphicspath{{../}}
\author{Cezary Wieczorkowski}
\date{\today}
\title{Wstęp}
\hypersetup{
 pdfauthor={Cezary Wieczorkowski},
 pdftitle={Wstęp},
 pdfkeywords={},
 pdfsubject={},
 pdflang={Polish}}
\begin{document}


\section{Wstęp i cel pracy}

Elektronika cyfrowa obecna jest w prawie każdym nowoczesnym urządzeniu. Jest to jedna z ważniejszych dziedzin elektroniki 
która umożliwia rozwój nowoczesnych technologii które leżą u podstaw współczesnego społeczeństwa. W procesie projektowania układów cyfrowych
zawsze zachodzi potrzeba wymiany danych pomiędzy różnymi podzespołami układu. Jednym z sposobów realizacji przepływu danych jest zastosowanie
szyny danych. Jest to zespół linii (przewodów) służących do przesyłania danych pomiędzy nadajnikami oraz odbiornikami do niej podłączonymi. 
\par
Celem pracy jest budowa stanowiska laboratoryjnego do realizacji ćwiczeń z sterowaniem szyną danych w laboratorium techniki cyfrowej.
To ćwiczenie ma za zadanie zapoznać studentów z sposobami realizacji komunikacji między różnymi układami cyfrowymi przy pomocy
wspólnej magistrali danych. Rozwiązania tego typu są powszechnie stosowane w nowoczesnych układach cyfrowych takich jak mikrokontrolery oraz
mikroprocesory w celu wymiany danych pomiędzy różnymi podzespołami danego układu. Zrozumienie mechanizmów działania systemu opartego o 
magistralę danych pomoże studentom w zrozumienia zasad działania bardziej skomplikowanych układów cyfrowych.

\end{document}