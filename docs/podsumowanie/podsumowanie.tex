\documentclass[../main.tex]{subfiles}
\usepackage[utf8]{inputenc}
\usepackage[T1]{fontenc}
\usepackage{graphicx}
\usepackage{longtable}
\usepackage{wrapfig}
\usepackage{rotating}
\usepackage[normalem]{ulem}
\usepackage{amsmath}
\usepackage{amssymb}
\usepackage{capt-of}
\usepackage{hyperref}
\usepackage{float}
\graphicspath{{../}}
\author{Cezary Wieczorkowski}
\date{\today}
\title{Podsumowanie}
\hypersetup{
 pdfauthor={Cezary Wieczorkowski},
 pdftitle={Podsumowanie},
 pdfkeywords={},
 pdfsubject={},
 pdflang={Polish}}
\begin{document}


\section{Podsumowanie}

Celem pracy była budowa stanowiska laboratoryjnego do realizacji ćwiczeń z sterowaniem szyną danych w laboratorium techniki cyfrowej.
\par
W ramach pracy dokonano krytycznej analizy istniejącego stanowiska laboratoryjnego. Przeanalizowano stanowisko pod kątem jego funkcjonalności,
architektury sprzętowej oraz interfejsu użytkownika.
\par
Na podstawie wyników analizy sporządzono listę wymagań dla nowego stanowiska. Opracowano architekturę sprzętową oraz programową nowego stanowiska.
Dokonano wyboru elementów sprzętowych oraz oprogramowania niezbędnego do realizacji nowego stanowiska.
\par
Zaprojektowano oraz zbudowano prototyp części sprzętowej nowego stanowiska. Napisano oprogramowanie implementujące funkcjonalności nowego stanowiska.
\par
Przeprowadzono testy prototypu stanowiska w celu weryfikacji poprawności działania oraz spełnienia wymagań postawionych przed nowym stanowiskiem.
\par
Dalsze prace nad projektem powinny skupić się na przeniesieniu konstrukcji prototypu na płytkę drukowaną. Należy również zaprojektować i wykonać 
obudowę stanowiska. Alternatywnie można rozważyć adaptacje obudowy istniejącego stanowiska do nowego projektu. Do oprogramowania stanowiska 
można dodać funkcję zapisu pamięci na podstawie pliku tekstowego przygotowanego przez użytkownika. Z pewnością usprawniło by to proces
wykonywania ćwiczeń na stanowisku.

\end{document}